\newpage
%-------------------------------------------------------------------------------
\section{\File{atlasjournal.sty}}

Turn on including these definitions with the option \Option{journal=true} and off with the option \Option{journal=false}.

\begin{xtabular}{ll}
\verb|\AcPA| & \AcPA \\
\verb|\ARevNS| & \ARevNS \\
\verb|\CPC| & \CPC \\
\verb|\EPJ| & \EPJ \\
\verb|\EPJC| & \EPJC \\
\verb|\FortP| & \FortP \\
\verb|\IJMP| & \IJMP \\
\verb|\JETP| & \JETP \\
\verb|\JETPL| & \JETPL \\
\verb|\JaFi| & \JaFi \\
\verb|\JHEP| & \JHEP \\
\verb|\JMP| & \JMP \\
\verb|\MPL| & \MPL \\
\verb|\NCim| & \NCim \\
\verb|\NIM| & \NIM \\
\verb|\NIMA| & \NIMA \\
\verb|\NP| & \NP \\
\verb|\NPB| & \NPB \\
\verb|\PL| & \PL \\
\verb|\PLB| & \PLB \\
\verb|\PR| & \PR \\
\verb|\PRC| & \PRC \\
\verb|\PRD| & \PRD \\
\verb|\PRL| & \PRL \\
\verb|\PRep| & \PRep \\
\verb|\RMP| & \RMP \\
\verb|\ZfP| & \ZfP \\
\verb|\collab| & \collab \\
\end{xtabular}



\newpage
%-------------------------------------------------------------------------------
\section{\File{atlasmisc.sty}}

Turn on including these definitions with the option \Option{misc=true} and off with the option \Option{misc=false}.

\begin{xtabular}{ll}
\verb|\pT| & \pT \\
\verb|\pt| & \pt \\
\verb|\ET| & \ET \\
\verb|\eT| & \eT \\
\verb|\et| & \et \\
\verb|\HT| & \HT \\
\verb|\pTsq| & \pTsq \\
\verb|\MET| & \MET \\
\verb|\met| & \met \\
\verb|\sumET| & \sumET \\
\verb|\EjetRec| & \EjetRec \\
\verb|\PjetRec| & \PjetRec \\
\verb|\EjetTru| & \EjetTru \\
\verb|\PjetTru| & \PjetTru \\
\verb|\EjetDM| & \EjetDM \\
\verb|\Rcone| & \Rcone \\
\verb|\abseta| & \abseta \\
\verb|\Ecm| & \Ecm \\
\verb|\rts| & \rts \\
\verb|\sqs| & \sqs \\
\verb|\Nevt| & \Nevt \\
\verb|\zvtx| & \zvtx \\
\verb|\dzero| & \dzero \\
\verb|\zzsth| & \zzsth \\
\verb|\RunOne| & \RunOne \\
\verb|\RunTwo| & \RunTwo \\
\verb|\RunThr| & \RunThr \\
\verb|\kt| & \kt \\
\verb|\antikt| & \antikt \\
\verb|\Antikt| & \Antikt \\
\verb|\pileup| & \pileup \\
\verb|\Pileup| & \Pileup \\
\verb|\btag| & \btag \\
\verb|\btagged| & \btagged \\
\verb|\bquark| & \bquark \\
\verb|\bquarks| & \bquarks \\
\verb|\bjet| & \bjet \\
\verb|\bjets| & \bjets \\
\verb|\mh| & \mh \\
\verb|\mW| & \mW \\
\verb|\mZ| & \mZ \\
\verb|\mH| & \mH \\
\verb|\ACERMC| & \ACERMC \\
\verb|\ALPGEN| & \ALPGEN \\
\verb|\COLLIER| & \COLLIER \\
\verb|\EVTGEN| & \EVTGEN \\
\verb|\GEANT| & \GEANT \\
\verb|\GGTOVV| & \GGTOVV \\
\verb|\GOSAM| & \GOSAM \\
\verb|\Herwigpp| & \Herwigpp \\
\verb|\HERWIGpp| & \HERWIGpp \\
\verb|\Herwig| & \Herwig \\
\verb|\HATHOR| & \HATHOR \\
\verb|\HERWIG| & \HERWIG \\
\verb|\JIMMY| & \JIMMY \\
\verb|\MADSPIN| & \MADSPIN \\
\verb|\MADGRAPH| & \MADGRAPH \\
\verb|\MEPSatLO| & \MEPSatLO \\
\verb|\MEPSatNLO| & \MEPSatNLO \\
\verb|\MGMCatNLO| & \MGMCatNLO \\
\verb|\MCatNLO| & \MCatNLO \\
\verb|\MINLO| & \MINLO \\
\verb|\AMCatNLO| & \AMCatNLO \\
\verb|\MCFM| & \MCFM \\
\verb|\METOP| & \METOP \\
\verb|\POWHEG| & \POWHEG \\
\verb|\POWHEGBOX| & \POWHEGBOX \\
\verb|\POWPYTHIA| & \POWPYTHIA \\
\verb|\PHOTOSPP| & \PHOTOSPP \\
\verb|\PROTOS| & \PROTOS \\
\verb|\PYTHIA| & \PYTHIA \\
\verb|\SHERPA| & \SHERPA \\
\verb|\VBFNLO| & \VBFNLO \\
\verb|\AUET| & \AUET \\
\verb|\AZNLO| & \AZNLO \\
\verb|\Comphep| & \Comphep \\
\verb|\FXFX| & \FXFX \\
\verb|\Monash| & \Monash \\
\verb|\NLOEWvirt| & \NLOEWvirt \\
\verb|\Perugia| & \Perugia \\
\verb|\Prospino| & \Prospino \\
\verb|\UEEE| & \UEEE \\
\verb|\LO| & \LO \\
\verb|\NLO| & \NLO \\
\verb|\NLL| & \NLL \\
\verb|\NNLO| & \NNLO \\
\verb|\muF| & \muF \\
\verb|\muQ| & \muQ \\
\verb|\muR| & \muR \\
\verb|\hdamp| & \hdamp \\
\verb|\ra| & \ra \\
\verb|\la| & \la \\
\verb|\rarrow| & \rarrow \\
\verb|\larrow| & \larrow \\
\verb|\lapprox| & \lapprox \\
\verb|\rapprox| & \rapprox \\
\verb|\gam| & \gam \\
\verb|\stat| & \stat \\
\verb|\syst| & \syst \\
\verb|\radlength| & \radlength \\
\verb|\StoB| & \StoB \\
\verb|\alphas| & \alphas \\
\verb|\NF| & \NF \\
\verb|\NC| & \NC \\
\verb|\CF| & \CF \\
\verb|\CA| & \CA \\
\verb|\TF| & \TF \\
\verb|\Lms| & \Lms \\
\verb|\Lmsfive| & \Lmsfive \\
\verb|\kperp| & \kperp \\
\verb|\Vcb| & \Vcb \\
\verb|\Vub| & \Vub \\
\verb|\Vtd| & \Vtd \\
\verb|\Vts| & \Vts \\
\verb|\Vtb| & \Vtb \\
\verb|\Vcs| & \Vcs \\
\verb|\Vud| & \Vud \\
\verb|\Vus| & \Vus \\
\verb|\Vcd| & \Vcd \\
\end{xtabular}


\noindent A length \Macro{figwidth} is defined that is \SI{2}{\cm} smaller than \Macro{textwidth}.

\noindent Most Monte Carlo generators also have a form with a suffix \enquote{V}
that allows you to include the version, e.g.
\verb|\PYTHIAV{8}| to produce \PYTHIAV{8} or
\verb|\PYTHIAV{8 (v8.160)}| to produce \PYTHIAV{8 (v8.160)}.

\noindent A generic macro \verb|\twomass| is defined, so that for example
\verb|\twomass{\mu}{\mu}| produces \twomass{\mu}{\mu} and \verb|\twomass{\mu}{e}| produces \twomass{\mu}{e}.

A macro \verb|\dk| is also defined which makes it easier to write down decay chains.
For example
\begin{verbatim}
\[\eqalign{a \to & b+c\\
   & \dk & e+f \\
   && \dk g+h}
\]
\end{verbatim}
produces
\[\eqalign{a \to & b+c\cr
   & \dk & e+f \cr
   && \dk g+h}
\]
Note that \Macro{eqalign} is also redefined in this package so that \Macro{dk} works.

The following macro names have been changed:\\
\verb|\ptsq| \(\to\) \verb|\pTsq|.


\newpage
%-------------------------------------------------------------------------------
\section{\File{atlasxref.sty}}

Turn on including these definitions with the option \Option{xref=true} and off with the option \Option{xref=false}.

%------------------------------------------------------------------------------
% Useful abbreviations in text for cross references.
% The abbreviations can be adjusted depending on the journal.
% An alternative is to use the cleveref package.
%
% Note that this file can be overwritten when atlaslatex is updated.
%
% Copyright (C) 2002-2020 CERN for the benefit of the ATLAS collaboration.
%------------------------------------------------------------------------------

\newcommand*{\App}[1]{Appendix~#1\xspace}
\newcommand*{\Eqn}[1]{Eq.~#1\xspace}
\newcommand*{\Fig}[1]{Figure~#1\xspace}
\newcommand*{\Refn}[1]{Ref.~#1\xspace}
\newcommand*{\Sect}[1]{Section~#1\xspace}
\newcommand*{\Tab}[1]{Table~#1\xspace}
\newcommand*{\Apps}[2]{Appendices~#1 and #2\xspace}
\newcommand*{\Eqns}[2]{Eqs.~#1 and #2\xspace}
\newcommand*{\Figs}[2]{Figures~#1 and #2\xspace}
\newcommand*{\Refns}[2]{Refs.~#1 and #2\xspace}
\newcommand*{\Sects}[2]{Sections~#1 and #2\xspace}
\newcommand*{\Tabs}[2]{Tables~#1 and #2\xspace}
\newcommand*{\Apprange}[2]{Appendices~#1--#2\xspace}
\newcommand*{\Eqnrange}[2]{Eqs.~#1--#2\xspace}
\newcommand*{\Figrange}[2]{Figures~#1--#2\xspace}
\newcommand*{\Refrange}[2]{Refs.~#1--#2\xspace}
\newcommand*{\Sectrange}[2]{Sections~#1--#2\xspace}
\newcommand*{\Tabrange}[2]{Tables~#1--#2\xspace}


\noindent The following macros with arguments are also defined:
\begin{xtabular}{ll}
\verb|\App{1}|  & \App{1}\\
\verb|\Eqn{1}|  & \Eqn{1}\\
\verb|\Fig{1}|  & \Fig{1}\\
\verb|\Refn{1}|  & \Refn{1}\\
\verb|\Sect{1}| & \Sect{1}\\
\verb|\Tab{1}|  & \Tab{1}\\
\verb|\Apps{1}{4}| & \Apps{1}{4} \\
\verb|\Eqns{1}{4}| & \Eqns{1}{4} \\
\verb|\Figs{1}{4}| & \Figs{1}{4} \\
\verb|\Refns{1}{4}| & \Refns{1}{4} \\
\verb|\Sects{1}{4}| & \Sects{1}{4} \\
\verb|\Tabs{1}{4}| & \Tabs{1}{4} \\
\verb|\Apprange{1}{4}| & \Apprange{1}{4} \\
\verb|\Eqnrange{1}{4}| & \Eqnrange{1}{4} \\
\verb|\Figrange{1}{4}| & \Figrange{1}{4} \\
\verb|\Refrange{1}{4}| & \Refrange{1}{4} \\
\verb|\Sectrange{1}{4}| & \Sectrange{1}{4} \\
\verb|\Tabrange{1}{4}| & \Tabrange{1}{4}
\end{xtabular}

The idea is that you can adapt these definitions according to your own preferences (or those of a journal).
Note that the macros \Macro{Ref} and \Macro{Refs} were renamed to \Macro{Refn} and \Macro{Refns}
in \Package{atlaslatex} 08-00-00, as \Macro{Ref} is now defined in the \Package{hyperref} package.


\newpage
%-------------------------------------------------------------------------------
\section{\File{atlasbsm.sty}}

Turn on including these definitions with the option \Option{BSM} and off with the option \Option{BSM=false}.

The macro \Macro{susy} simply puts a tilde (\(\tilde{\ }\)) over its argument,
e.g.\ \verb|\susy{q}| produces \susy{q}.

For \susy{q}, \susy{t}, \susy{b}, \slepton, \sel, \smu and
\stau, L and R states are defined; for stop, sbottom and stau also the
light (1) and heavy (2) states.
There are four neutralinos and two charginos defined,
the index number unfortunately needs to be written out completely.
For the charginos the last letter(s) indicate(s) the charge:
\enquote{p} for \(+\), \enquote{m} for \(-\), and \enquote{pm} for \(\pm\).

\begin{xtabular}{ll}
\verb|\Azero| & \Azero \\
\verb|\hzero| & \hzero \\
\verb|\Hzero| & \Hzero \\
\verb|\Hboson| & \Hboson \\
\verb|\Hplus| & \Hplus \\
\verb|\Hminus| & \Hminus \\
\verb|\Hpm| & \Hpm \\
\verb|\Hmp| & \Hmp \\
\verb|\ggino| & \ggino \\
\verb|\chinop| & \chinop \\
\verb|\chinom| & \chinom \\
\verb|\chinopm| & \chinopm \\
\verb|\chinomp| & \chinomp \\
\verb|\chinoonep| & \chinoonep \\
\verb|\chinoonem| & \chinoonem \\
\verb|\chinoonepm| & \chinoonepm \\
\verb|\chinotwop| & \chinotwop \\
\verb|\chinotwom| & \chinotwom \\
\verb|\chinotwopm| & \chinotwopm \\
\verb|\nino| & \nino \\
\verb|\ninoone| & \ninoone \\
\verb|\ninotwo| & \ninotwo \\
\verb|\ninothree| & \ninothree \\
\verb|\ninofour| & \ninofour \\
\verb|\gravino| & \gravino \\
\verb|\Zprime| & \Zprime \\
\verb|\Zstar| & \Zstar \\
\verb|\squark| & \squark \\
\verb|\squarkL| & \squarkL \\
\verb|\squarkR| & \squarkR \\
\verb|\gluino| & \gluino \\
\verb|\stop| & \stop \\
\verb|\stopone| & \stopone \\
\verb|\stoptwo| & \stoptwo \\
\verb|\stopL| & \stopL \\
\verb|\stopR| & \stopR \\
\verb|\sbottom| & \sbottom \\
\verb|\sbottomone| & \sbottomone \\
\verb|\sbottomtwo| & \sbottomtwo \\
\verb|\sbottomL| & \sbottomL \\
\verb|\sbottomR| & \sbottomR \\
\verb|\slepton| & \slepton \\
\verb|\sleptonL| & \sleptonL \\
\verb|\sleptonR| & \sleptonR \\
\verb|\sel| & \sel \\
\verb|\selL| & \selL \\
\verb|\selR| & \selR \\
\verb|\smu| & \smu \\
\verb|\smuL| & \smuL \\
\verb|\smuR| & \smuR \\
\verb|\stau| & \stau \\
\verb|\stauL| & \stauL \\
\verb|\stauR| & \stauR \\
\verb|\stauone| & \stauone \\
\verb|\stautwo| & \stautwo \\
\verb|\snu| & \snu \\
\end{xtabular}



\newpage
%-------------------------------------------------------------------------------
\section{\File{atlasheavyion.sty}}

Turn on including these definitions with the option \Option{hion=true} and off with the option \Option{hion=false}.
The heavy ion definitions use the package \Package{mhchem} to help with the formatting of chemical elements.
This package is included by \File{atlasheavyion.sty}.

%-------------------------------------------------------------------------------
% Collection of heavy iondefinitions, typically not included in the other style files.
% Include with hion option in atlasphysics.sty.
% Not included by default.
% Also needs atlasmisc.sty (option misc).
% Compiled by Sasha Milov.
% Adapted for atlaslatex by Ian Brock.
%
% Note that this file can be overwritten when atlaslatex is updated.
%
% Copyright (C) 2002-2020 CERN for the benefit of the ATLAS collaboration.
%-------------------------------------------------------------------------------

% Package used for chemical elements
\RequirePackage[version=3]{mhchem}

% +------------------------------------+
% |                                    |
% |  System related notations          |
% |                                    |
% +------------------------------------+
\newcommand*{\NucNuc}{\ce{A}+\ce{A}\xspace}

\newcommand*{\nn}{\ensuremath{nn}\xspace}
%\newcommand*{\pp}{\ensuremath{pp}\xspace}
\newcommand*{\pn}{\ensuremath{pn}\xspace}
\newcommand*{\np}{\ensuremath{np}\xspace}

\newcommand*{\PbPb}{\ce{Pb}+\ce{Pb}\xspace}
\newcommand*{\AuAu}{\ce{Au}+\ce{Au}\xspace}
\newcommand*{\CuCu}{\ce{Cu}+\ce{Cu}\xspace}

\providecommand*{\pA}{\ensuremath{p}+\ce{A}\xspace}
\newcommand*{\pNuc}{\pA\xspace}
\newcommand*{\pdA}{\ensuremath{p}/\ensuremath{d}+\ce{A}\xspace}
\newcommand*{\dAu}{\ensuremath{d}+\ce{Au}\xspace}
\newcommand*{\pPb}{\ensuremath{p}+\ce{Pb}\xspace}

% +--------------------------------------+
% |                                      |
% |  Centrality related notations        |
% |                                      |
% +--------------------------------------+
\newcommand*{\Npart}{\ensuremath{N_{\text{part}}}\xspace}
\newcommand*{\avgNpart}{\ensuremath{\langle\Npart\rangle}\xspace}

\newcommand*{\Ncoll}{\ensuremath{N_{\text{coll}}}\xspace}
\newcommand*{\avgNcoll}{\ensuremath{\langle\Ncoll\rangle}\xspace}

\newcommand*{\TA}{\ensuremath{T_{\ce{A}}}\xspace}
\newcommand*{\avgTA}{\ensuremath{\langle\TA\rangle}\xspace}

\newcommand*{\TPb}{\ensuremath{T_{\ce{Pb}}}\xspace}
\newcommand*{\avgTPb}{\ensuremath{\langle\TPb\rangle}\xspace}

\newcommand*{\TAA}{\ensuremath{T_{\text{AA}}}\xspace}
\newcommand*{\avgTAA}{\ensuremath{\langle\TAA\rangle}\xspace}

\newcommand{\TAB}{\ensuremath{T_{\text{AB}}}\xspace}
\newcommand{\avgTAB}{\ensuremath{\langle\TAB\rangle}\xspace}

\newcommand*{\TpPb}{\ensuremath{T_{p\ce{Pb}}}\xspace}
\newcommand*{\avgTpPb}{\ensuremath{\langle\TpPb\rangle}\xspace}

\newcommand{\Gl}{Glauber\xspace}
\newcommand{\GG}{Glauber--Gribov\xspace}

% +--------------------------------------+
% |                                      |
% |  C.M. energy related notations       |
% |                                      |
% +--------------------------------------+
\newcommand*{\sqn}{\ensuremath{\sqrt{s_{_\text{NN}}}}\xspace}
\newcommand{\lns}{\ensuremath{\ln(\kern -0.2em\sqrt{s})}\xspace}

% +--------------------------------------+
% |                                      |
% |  Some useful parameters              |
% |                                      |
% +--------------------------------------+
\newcommand*{\sumETPb}{\ensuremath{\Sigma E_{\text{T}}^{\ce{Pb}}}\xspace}
\newcommand*{\sumETp}{\ensuremath{\Sigma E_{\text{T}}^{p}}\xspace}
\newcommand*{\sumETA}{\ensuremath{\Sigma E_{\text{T}}^{\ce{A}}}\xspace}

% +--------------------------------------+
% |                                      |
% |  Some useful constructions           |
% |                                      |
% +--------------------------------------+
\newcommand*{\RAA}{\ensuremath{R_{\ce{AA}}}\xspace}
\newcommand*{\RCP}{\ensuremath{R_{\text{CP}}}\xspace}
\newcommand*{\RpA}{\ensuremath{R_{p\ce{A}}}\xspace}

\newcommand*{\RpPb}{\ensuremath{R_{p\ce{Pb}}}\xspace}

% Different differential symbols in American and British English
\iflanguage{USenglish}{%
  \providecommand*{\dif}{\ensuremath{d}}
}{%
  \providecommand*{\dif}{\ensuremath{\mathrm{d}}}
}
\newcommand*{\dNchdeta}{\ensuremath{\dif N_{\text{ch}}/\dif \eta}\xspace}
\newcommand*{\dNevtdET}{\ensuremath{\dif N_{\text{evt}}/\dif \ET}\xspace}

% +--------------------------------------+
% |                                      |
% |  Framework transforms                |
% |                                      |
% +--------------------------------------+
\newcommand*{\ystar}{\ensuremath{y^{*}}\xspace}
\newcommand*{\ycms}{\ensuremath{y_\text{CM}}\xspace}
\newcommand*{\ygappb}{\ensuremath{\Delta \eta_{\text{gap}}^{\ce{Pb}}}\xspace}
\newcommand*{\ygapp}{\ensuremath{\Delta \eta_{\text{gap}}^{p}}\xspace}
\newcommand*{\fgap}{\ensuremath{f_{\text{gap}}}\xspace}


%The following symbols were removed or modified with respect to the original submission
%\input{atlasheavyion-mod}


\newpage
%-------------------------------------------------------------------------------
\section{\File{atlasjetetmiss.sty}}

Turn on including these definitions with the option \Option{jetetmiss=true} and off with the option \Option{jetetmiss=false}.

\begin{xtabular}{ll}
\verb|\topo| & \topo \\
\verb|\Topo| & \Topo \\
\verb|\topos| & \topos \\
\verb|\Topos| & \Topos \\
\verb|\insitu| & \insitu \\
\verb|\Insitu| & \Insitu \\
\verb|\LS| & \LS \\
\verb|\NLOjet| & \NLOjet \\
\verb|\Fastjet| & \Fastjet \\
\verb|\TwoToTwo| & \TwoToTwo \\
\verb|\largeR| & \largeR \\
\verb|\LargeR| & \LargeR \\
\verb|\akt| & \akt \\
\verb|\Akt| & \Akt \\
\verb|\AKT| & \AKT \\
\verb|\AKTFat| & \AKTFat \\
\verb|\AKTPrune| & \AKTPrune \\
\verb|\AKTFilt| & \AKTFilt \\
\verb|\KTSix| & \KTSix \\
\verb|\ca| & \ca \\
\verb|\CamKt| & \CamKt \\
\verb|\CASix| & \CASix \\
\verb|\CAFat| & \CAFat \\
\verb|\CAPrune| & \CAPrune \\
\verb|\CAFilt| & \CAFilt \\
\verb|\htt| & \htt \\
\verb|\mcut| & \mcut \\
\verb|\Nfilt| & \Nfilt \\
\verb|\Rfilt| & \Rfilt \\
\verb|\ymin| & \ymin \\
\verb|\fcut| & \fcut \\
\verb|\Rsub| & \Rsub \\
\verb|\mufrac| & \mufrac \\
\verb|\Rcut| & \Rcut \\
\verb|\zcut| & \zcut \\
\verb|\ftile| & \ftile \\
\verb|\fem| & \fem \\
\verb|\fpres| & \fpres \\
\verb|\fhec| & \fhec \\
\verb|\ffcal| & \ffcal \\
\verb|\central| & \central \\
\verb|\ecap| & \ecap \\
\verb|\forward| & \forward \\
\verb|\Npv| & \Npv \\
\verb|\Nref| & \Nref \\
\verb|\Navg| & \Navg \\
\verb|\avgmu| & \avgmu \\
\verb|\JES| & \JES \\
\verb|\JMS| & \JMS \\
\verb|\EMJES| & \EMJES \\
\verb|\GCWJES| & \GCWJES \\
\verb|\LCWJES| & \LCWJES \\
\verb|\EM| & \EM \\
\verb|\GCW| & \GCW \\
\verb|\LCW| & \LCW \\
\verb|\GSL| & \GSL \\
\verb|\GS| & \GS \\
\verb|\MTF| & \MTF \\
\verb|\MPF| & \MPF \\
\verb|\Njet| & \Njet \\
\verb|\njet| & \njet \\
\verb|\ETjet| & \ETjet \\
\verb|\etjet| & \etjet \\
\verb|\pTavg| & \pTavg \\
\verb|\ptavg| & \ptavg \\
\verb|\pTjet| & \pTjet \\
\verb|\ptjet| & \ptjet \\
\verb|\pTcorr| & \pTcorr \\
\verb|\ptcorr| & \ptcorr \\
\verb|\pTjeti| & \pTjeti \\
\verb|\ptjeti| & \ptjeti \\
\verb|\pTrecoil| & \pTrecoil \\
\verb|\ptrecoil| & \ptrecoil \\
\verb|\pTleading| & \pTleading \\
\verb|\ptleading| & \ptleading \\
\verb|\pTjetEM| & \pTjetEM \\
\verb|\ptjetEM| & \ptjetEM \\
\verb|\pThat| & \pThat \\
\verb|\pthat| & \pthat \\
\verb|\pTprobe| & \pTprobe \\
\verb|\ptprobe| & \ptprobe \\
\verb|\pTref| & \pTref \\
\verb|\ptref| & \ptref \\
\verb|\pToff| & \pToff \\
\verb|\ptoff| & \ptoff \\
\verb|\pToffjet| & \pToffjet \\
\verb|\ptoffjet| & \ptoffjet \\
\verb|\pTZ| & \pTZ \\
\verb|\ptZ| & \ptZ \\
\verb|\pTtrue| & \pTtrue \\
\verb|\pttrue| & \pttrue \\
\verb|\pTtruth| & \pTtruth \\
\verb|\pttruth| & \pttruth \\
\verb|\pTreco| & \pTreco \\
\verb|\ptreco| & \ptreco \\
\verb|\pTtrk| & \pTtrk \\
\verb|\pttrk| & \pttrk \\
\verb|\ptrk| & \ptrk \\
\verb|\pTtrkjet| & \pTtrkjet \\
\verb|\pttrkjet| & \pttrkjet \\
\verb|\ntrk| & \ntrk \\
\verb|\EoverP| & \EoverP \\
\verb|\Etrue| & \Etrue \\
\verb|\Etruth| & \Etruth \\
\verb|\Ecalo| & \Ecalo \\
\verb|\EcaloEM| & \EcaloEM \\
\verb|\asym| & \asym \\
\verb|\Response| & \Response \\
\verb|\Rcalo| & \Rcalo \\
\verb|\Rcalom| & \Rcalom \\
\verb|\RcaloEM| & \RcaloEM \\
\verb|\RMPF| & \RMPF \\
\verb|\EcaloCALIB| & \EcaloCALIB \\
\verb|\RcaloCALIB| & \RcaloCALIB \\
\verb|\EcaloEMJES| & \EcaloEMJES \\
\verb|\RcaloEMJES| & \RcaloEMJES \\
\verb|\EcaloGCWJES| & \EcaloGCWJES \\
\verb|\RcaloGCWJES| & \RcaloGCWJES \\
\verb|\EcaloLCWJES| & \EcaloLCWJES \\
\verb|\RcaloLCWJES| & \RcaloLCWJES \\
\verb|\Rtrack| & \Rtrack \\
\verb|\rtrk| & \rtrk \\
\verb|\Rtrk| & \Rtrk \\
\verb|\rtrackjet| & \rtrackjet \\
\verb|\rtrackjetiso| & \rtrackjetiso \\
\verb|\rtrackjetnoniso| & \rtrackjetnoniso \\
\verb|\rtrackjetisoratio| & \rtrackjetisoratio \\
\verb|\gammajet| & \gammajet \\
\verb|\deltaphijetgamma| & \deltaphijetgamma \\
\verb|\rapjet| & \rapjet \\
\verb|\etajet| & \etajet \\
\verb|\phijet| & \phijet \\
\verb|\etadet| & \etadet \\
\verb|\etatrk| & \etatrk \\
\verb|\Rmin| & \Rmin \\
\verb|\DeltaR| & \DeltaR \\
\verb|\DetaDphi| & \DetaDphi \\
\verb|\Deta| & \Deta \\
\verb|\Drap| & \Drap \\
\verb|\DetaOneTwo| & \DetaOneTwo \\
\verb|\DyDphi| & \DyDphi \\
\verb|\DeltaRdef| & \DeltaRdef \\
\verb|\DeltaRydef| & \DeltaRydef \\
\verb|\DeltaRtrk| & \DeltaRtrk \\
\verb|\JVF| & \JVF \\
\verb|\cJVF| & \cJVF \\
\verb|\RpT| & \RpT \\
\verb|\JVT| & \JVT \\
\verb|\ghostpt| & \ghostpt \\
\verb|\ghostptavg| & \ghostptavg \\
\verb|\ghostfm| & \ghostfm \\
\verb|\ghostfmi| & \ghostfmi \\
\verb|\ghostdensity| & \ghostdensity \\
\verb|\ghostrho| & \ghostrho \\
\verb|\Aghost| & \Aghost \\
\verb|\Amu| & \Amu \\
\verb|\Amui| & \Amui \\
\verb|\jetarea| & \jetarea \\
\verb|\jetareafm| & \jetareafm \\
\verb|\jetareai| & \jetareai \\
\verb|\Rkt| & \Rkt \\
\verb|\pTmuslope| & \pTmuslope \\
\verb|\ptmuslope| & \ptmuslope \\
\verb|\pTnpvslope| & \pTnpvslope \\
\verb|\ptnpvslope| & \ptnpvslope \\
\verb|\pTmuunc| & \pTmuunc \\
\verb|\ptmuunc| & \ptmuunc \\
\verb|\pTnpvunc| & \pTnpvunc \\
\verb|\ptnpvunc| & \ptnpvunc \\
\verb|\sumPt| & \sumPt \\
\verb|\sumpt| & \sumpt \\
\verb|\sumpTtrk| & \sumpTtrk \\
\verb|\sumpttrk| & \sumpttrk \\
\verb|\nPUtrk| & \nPUtrk \\
\verb|\mjet| & \mjet \\
\verb|\mlead| & \mlead \\
\verb|\mleadavg| & \mleadavg \\
\verb|\Mjet| & \Mjet \\
\verb|\massjet| & \massjet \\
\verb|\masscorr| & \masscorr \\
\verb|\mthresh| & \mthresh \\
\verb|\mjetavg| & \mjetavg \\
\verb|\masstrkjet| & \masstrkjet \\
\verb|\width| & \width \\
\verb|\wcalo| & \wcalo \\
\verb|\wtrk| & \wtrk \\
\verb|\shapeV| & \shapeV \\
\verb|\pTsubjet| & \pTsubjet \\
\verb|\ptsubjet| & \ptsubjet \\
\verb|\sjone| & \sjone \\
\verb|\sjtwo| & \sjtwo \\
\verb|\msubjone| & \msubjone \\
\verb|\msubjtwo| & \msubjtwo \\
\verb|\pTsubji| & \pTsubji \\
\verb|\ptsubji| & \ptsubji \\
\verb|\pTsubjone| & \pTsubjone \\
\verb|\ptsubjone| & \ptsubjone \\
\verb|\pTsubjtwo| & \pTsubjtwo \\
\verb|\ptsubjtwo| & \ptsubjtwo \\
\verb|\Rsubjets| & \Rsubjets \\
\verb|\DRsubjets| & \DRsubjets \\
\verb|\yij| & \yij \\
\verb|\dcut| & \dcut \\
\verb|\dmin| & \dmin \\
\verb|\dij| & \dij \\
\verb|\Dij| & \Dij \\
\verb|\Donetwo| & \Donetwo \\
\verb|\Dtwothr| & \Dtwothr \\
\verb|\yonetwo| & \yonetwo \\
\verb|\ytwothr| & \ytwothr \\
\verb|\yonetwoDef| & \yonetwoDef \\
\verb|\ytwothrDef| & \ytwothrDef \\
\verb|\xj| & \xj \\
\verb|\jetFunc| & \jetFunc \\
\verb|\tauone| & \tauone \\
\verb|\tautwo| & \tautwo \\
\verb|\tauthr| & \tauthr \\
\verb|\tauN| & \tauN \\
\verb|\tautwoone| & \tautwoone \\
\verb|\tauthrtwo| & \tauthrtwo \\
\verb|\dip| & \dip \\
\verb|\diponetwo| & \diponetwo \\
\verb|\diptwothr| & \diptwothr \\
\verb|\diponethr| & \diponethr \\
\verb|\mtaSup| & \mtaSup \\
\verb|\mcalo| & \mcalo \\
\verb|\mcomb| & \mcomb \\
\verb|\ECFOne| & \ECFOne \\
\verb|\ECFTwo| & \ECFTwo \\
\verb|\ECFThr| & \ECFThr \\
\verb|\ECFThrNorm| & \ECFThrNorm \\
\verb|\DTwo| & \DTwo \\
\verb|\CTwo| & \CTwo \\
\verb|\FoxWolfRatio| & \FoxWolfRatio \\
\verb|\PlanarFlow| & \PlanarFlow \\
\verb|\Angularity| & \Angularity \\
\verb|\Aplanarity| & \Aplanarity \\
\verb|\KtDR| & \KtDR \\
\verb|\Qw| & \Qw \\
\verb|\NConst| & \NConst \\
\end{xtabular}


\noindent The macro \Macro{etaRange} produces what you would expect:
\verb|\etaRange{-2.5}{+2.5}| produces \etaRange{-2.5}{+2.5} while
\verb|\AetaRange{1.0}| produces \AetaRange{1.0}.
The macro \Macro{avg} can be used for average values:
\verb|\avg{\mu}| produces \avg{\mu}.


\newpage
%-------------------------------------------------------------------------------
\section{\File{atlasmath.sty}}

Turn on including these definitions with the option \Option{math=true} and off with the option \Option{math=false}.

\begin{xtabular}{ll}
\verb|\boxsq| & \boxsq \\
\verb|\grad| & \grad \\
\end{xtabular}


\noindent The macro \Macro{spinor} is also defined.
\verb|\spinor{u}| produces \spinor{u}.


\newpage
%-------------------------------------------------------------------------------
\section{\File{atlasother.sty}}

Turn on including these definitions with the option \Option{other} and off with the option \Option{other=false}.

%-------------------------------------------------------------------------------
% Other definitions that used to be in atlasphysics.sty.
% Include with other option in atlasphysics.sty.
% Not included by default.
% Also needs atlasmisc.sty (option misc) and
% Also needs atlasparticle.sty (option particle).
%
% Note that this file can be overwritten when atlaslatex is updated.
%
% Copyright (C) 2002-2020 CERN for the benefit of the ATLAS collaboration.
%-------------------------------------------------------------------------------

\chardef\letterchar=11
\chardef\otherchar=12
\chardef\eolinechar=5

\newcommand*{\etpt}{\ensuremath{1/p_{\text{T}} - 1/E_{\text{T}}}\xspace}
\newcommand*{\etptsig}{\ensuremath{(1/p_{\text{T}} - 1/E_{\text{T}})/(\sigma(1/p_{\text{T}}))}\xspace}
\newcommand*{\begL}{\ensuremath{\SI{E31}{\per\cm\squared\per\second}}\xspace}
\newcommand*{\lowL}{\ensuremath{\SI{E33}{\per\cm\squared\per\second}}\xspace}
\newcommand*{\highL}{\ensuremath{\SI{E34}{\per\cm\squared\per\second}}\xspace}

\newcommand*{\Epsb}{\ensuremath{\epsilon_{b}}\xspace}
\newcommand*{\Epsc}{\ensuremath{\epsilon_{c}}\xspace} % Subscript italic not roman (EE)

% Conflicts with \mA in siunitx version 1
\providecommand*{\mA}{\ensuremath{m_{A}}\xspace}

% +--------------------------------------------------------------------+
% |                                                                    |
% |  sin2thetaW m_W m_Z etc.                                           |
% |                                                                    |
% +--------------------------------------------------------------------+
\newcommand*{\Mtau}{\ensuremath{m_{\tau}}\xspace}
\newcommand*{\swsq}{\ensuremath{\sin^2\!\theta_{\text{W}}}\xspace}
\newcommand*{\swel}{\ensuremath{\sin^2\!\theta_{\text{eff}}^{\text{lept}}}\xspace}
\newcommand*{\swsqb}{\ensuremath{\sin^2\!\overline{\theta}_{\text{W}}}\xspace}
\newcommand*{\swsqon}{\ensuremath{\swsq\equiv 1-\mW^2/\mZ^2}\xspace}
\newcommand*{\gv}{\ensuremath{g_{\text{V}}}\xspace}
\newcommand*{\ga}{\ensuremath{g_{\text{A}}}\xspace}
\newcommand*{\gvbar}{\ensuremath{\bar{g}_\text{V}}\xspace}
\newcommand*{\gabar}{\ensuremath{\bar{g}_\text{A}}\xspace}

% +--------------------------------------------------------------------+
% |                                                                    |
% |  Useful Z0 type stuff    Gammas, asymmetries                       |
% |                                                                    |
% +--------------------------------------------------------------------+
\newcommand*{\Zzv}{\ensuremath{Z^{\textstyle *}}\xspace}
\newcommand*{\Abb}{\ensuremath{A_{\bbbar}}\xspace}
\newcommand*{\Acc}{\ensuremath{A_{\ccbar}}\xspace}
\newcommand*{\Aqq}{\ensuremath{A_{\qqbar}}\xspace}
\newcommand*{\Afb}{\ensuremath{A_{{\text{FB}}}}\xspace}
\newcommand*{\GZ}{\ensuremath{\Gamma_{Z}}\xspace}
\newcommand*{\GW}{\ensuremath{\Gamma_{W}}\xspace}
\newcommand*{\GH}{\ensuremath{\Gamma_{H}}\xspace}
\newcommand*{\GamHad}{\ensuremath{\Gamma_{\text{had}}}\xspace}
\newcommand*{\Gbb}{\ensuremath{\Gamma_{\bbbar}}\xspace}
\newcommand*{\Rbb}{\ensuremath{R_{\bbbar}}\xspace}
\newcommand*{\Gcc}{\ensuremath{\Gamma_{\ccbar}}\xspace}
\newcommand*{\Gvis}{\ensuremath{\Gamma_{\text{vis}}}\xspace}
\newcommand*{\Ginv}{\ensuremath{\Gamma_{\text{inv}}}\xspace}

% +--------------------------------------------------------------------+
% |                                                                    |
% |  Notation for the P-lines: \nspj211 -->  2 1P1, etc.               |
% |                                                                    |
% +--------------------------------------------------------------------+
\def\nsPj#1#2#3{\ensuremath{#1\,^{#2}\!P_{#3}}}%
\let\nspj=\nsPj
\def\nsSj#1#2#3{\ensuremath{#1\,^{#2}\!S_{#3}}}%
\let\nssj=\nsSj

